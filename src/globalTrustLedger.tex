\section{Global Trust Ledger}
  Observe that in the previous example, it was never said that $Alice$ thinks that $Y_1$
  is \textit{trustworthy}; $Alice$ only perceives $Y_1$ as \textit{valid}. The lack of a
  global transaction ledger makes it possible that $Charlie$, $Dave$ or $Emily$ have
  double spent the new transactions $Alice$ learned about, thus $Alice$ has no reason to
  assume that she exclusively owns the coins at the output of transaction $C$. In order to
  be insured in case a previous transaction has been double spent, there has to be some
  kind of global, public commitment from each of the players in question (or someone
  vouching for them) that a refund will be given in case fraud is committed.

  The global ledger consists of a directed graph of refund promises between players in
  case fraud is committed. In order for $Alice$ to be insured against the double spend of
  a transaction that lies before a UTXO she can spend, there must be a sufficiently heavy
  chain of insurances between her and each of the players that spent these preceding
  transactions. The following examples clarify the concept with regards to the
  trustworthiness of $Y_1$:
  \begin{center}
    \begin{dot2tex}[outputdir=dot2tex/, file=goodTrust]
      
\begin{tikzpicture}[>=latex,line join=bevel,]
%%
\begin{scope}
  \definecolor{strokecol}{rgb}{0.0,0.0,0.0};
  \pgfsetstrokecolor{strokecol}
  \draw (246.5bp,11.5bp) node {\textbf{Fig. \figlabel{fig:goodTrust}:} Trust graph that insures $Alice$     from $Y_1$};
\end{scope}
  \node (Dave) at (355.79bp,135.5bp) [draw,ellipse] {$Dave$};
  \node (Charile) at (355.79bp,219.5bp) [draw,ellipse] {$Charile$};
  \node (Emily) at (355.79bp,55.497bp) [draw,ellipse] {$Emily$};
  \node (Alice) at (130.71bp,135.5bp) [draw,ellipse] {$Alice$};
  \node (George) at (240.0bp,135.5bp) [draw,ellipse] {$George$};
  \draw [->] (George) ..controls (285.53bp,104.19bp) and (304.23bp,91.044bp)  .. (Emily);
  \draw (297.9bp,106.0bp) node {$1$};
  \draw [->] (George) ..controls (289.22bp,135.5bp) and (303.35bp,135.5bp)  .. (Dave);
  \draw (297.9bp,143.0bp) node {$2$};
  \draw [->] (Alice) ..controls (170.71bp,135.5bp) and (182.13bp,135.5bp)  .. (George);
  \draw (182.1bp,143.0bp) node {$3$};
  \draw [->] (George) ..controls (284.27bp,167.44bp) and (302.12bp,180.61bp)  .. (Charile);
  \draw (297.9bp,187.0bp) node {$3$};
%
\end{tikzpicture}

    \end{dot2tex}
  \end{center}
  In this example, if any of $Charlie$, $Dave$ or $Emily$ double spend their money and
  $Alice$ finds out, then she will post the proof of fraud to the global ledger and
  $George$ will automatically refund her with the money that he is insuring the fraudster.
  Observe that there is enough insurance to cover for any individual fraud.

  \begin{center}
    \begin{dot2tex}[outputdir=dot2tex/, file=badTrust]
      
\begin{tikzpicture}[>=latex,line join=bevel,]
%%
\begin{scope}
  \definecolor{strokecol}{rgb}{0.0,0.0,0.0};
  \pgfsetstrokecolor{strokecol}
  \draw (265.0bp,11.5bp) node {\textbf{Fig. \figlabel{fig:badTrust}:} Trust graph that does not     insure $Alice$ from $Y_1$};
\end{scope}
  \node (Dave) at (374.29bp,52.247bp) [draw,ellipse] {$Dave$};
  \node (Charile) at (374.29bp,136.25bp) [draw,ellipse] {$Charile$};
  \node (Alice) at (149.21bp,94.247bp) [draw,ellipse] {$Alice$};
  \node (George) at (258.5bp,94.247bp) [draw,ellipse] {$George$};
  \draw [->] (George) ..controls (306.88bp,76.762bp) and (322.96bp,70.825bp)  .. (Dave);
  \draw (316.4bp,82.747bp) node {$5$};
  \draw [->] (Alice) ..controls (189.21bp,94.247bp) and (200.63bp,94.247bp)  .. (George);
  \draw (200.6bp,101.75bp) node {$3$};
  \draw [->] (George) ..controls (304.72bp,110.94bp) and (318.0bp,115.84bp)  .. (Charile);
  \draw (316.4bp,123.75bp) node {$2$};
%
\end{tikzpicture}

    \end{dot2tex}
  \end{center}
  In this case however, $Alice$ stands entirely uninsured against fraud by $Emily$ and
  only partly against fraud by $Charlie$. She is well insured against fraud by $Dave$.
